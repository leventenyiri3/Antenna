\documentclass[12pt, a4paper]{article} % FIXED: Removed 'letterpaper'

% --- MARGINS ---
\usepackage[margin=2.5cm]{geometry} % This sets the borders

% --- PACKAGES ---
\usepackage{graphicx}
\usepackage{textcomp}
\usepackage[hungarian]{babel}
\usepackage[T1]{fontenc}
\usepackage[utf8]{inputenc}
\usepackage{caption}
\usepackage{subcaption}
\usepackage{csquotes}

% --- IMAGES PATH ---
\graphicspath{{./Pictures/}{./Pictures/koax}{./Pictures/erosito}{./Pictures/Kisjelu-erositok}{./Pictures/Nagyjelu-erositok}{./Pictures/Keverok-es-oszcillatorok}{./Pictures/Forrasztos}}

% --- BIBLIOGRAPHY ---
\usepackage[style=ieee]{biblatex}
\addbibresource{forrasok.bib}

% --- CODE LISTINGS ---
\usepackage{listings}
\usepackage{color}
\usepackage{amsmath}
\usepackage{placeins}

% --- COLORS & SETTINGS ---
\definecolor{dkgreen}{rgb}{0,0.6,0}
\definecolor{gray}{rgb}{0.5,0.5,0.5}
\definecolor{mauve}{rgb}{0.58,0,0.82}

\lstset{frame=tb,
  language=Bash,
  aboveskip=3mm,
  belowskip=3mm,
  showstringspaces=false,
  columns=flexible,
  basicstyle={\small\ttfamily},
  numbers=none,
  numberstyle=\tiny\color{gray},
  keywordstyle=\color{blue},
  commentstyle=\color{dkgreen},
  stringstyle=\color{mauve},
  breaklines=true,
  breakatwhitespace=true,
  tabsize=2
}

\title{Yagi antenna szimulációja}
\author{Nyiri Levente}
\date{2026 Január}

\begin{document}
\begin{figure}
  \centering
  \includegraphics[width=\textwidth]{BME.png}
\end{figure}
\selectlanguage{hungarian}
\maketitle

\newpage

\section{Bevezetés}

A Yagi-Uda antennát Shintaro Uda találta fel az 1920-as években, viszont eredményeit japánul dokumentálta, így a világ többi részén csak akkkor lett ismert, amikor Hidetsuru Yagi leírta az eredményeit angolul is. Annak ellenére, hogy kreditálta Uda munkáját, az antennára leggyakrabban csak Yagi antennaként hivatkoznak.

Ez egy antenna rendszer, amelyben egy dipólantennát vesznek körül parazita elemekkel, hogy jobb irányítottságot érjenek el.

Az elmúlt évtizedekben ez lett a legelterjedtebb antenna TV vételére VHF és UHF sávokon.

Én a feladatomhoz kaptam a Tanár Úrtól egy dokumentumot, amelyben különböző Yagi antenna dimenziók vannak megadva VHF sávon, ebből fogok kiindulni, egy 58MHz-s antennát fogok tervezni a K 52 16 8 27-es oszlop alapján.

\begin{figure}[ht]
    \centering
    \includegraphics[width=0.8\textwidth]{kathrein.jpg} 
    \caption{Antenna dimenziók}
    \label{fig:kathrein}
\end{figure}



\clearpage

\section{Elméleti ismeretek összefoglalása}

Egy Yagi antenna 3 részből áll: egy reflektorból (ritkán alkalmaznak többet, mivel jelentősen nem javítja a reflektivitást, jelen esetben pont 3 lesz), egy táplált félhullámú dipólból (a továbbiakban DE, mint driven element-ként fogok rá hivatkozni) és egy vagy több direktorból.
A felépítést a \ref{fig:yagi-konyv}. ábra mutatja. A DE leggyakrabban egy hajlított dipólus, én a szimulációban egyszerű vezetéket alkalmaztam.


\begin{figure}[ht]
    \centering
    \includegraphics[width=0.6\textwidth]{Yagi-konyv.png} 
    \caption{Yagi antenna általános felépítése \cite{balanis}}
    \label{fig:yagi-konyv}
\end{figure}

A reflektor valamivel nagyobb, mint a DE, a direktorok pedig valamivel kisebbek.

A sugárzási irány a direktorok irányába mutat, és minél több direktort használunk, annál jobb az irányító hatás, bár egy bizonyos szám fölött már elhanyagolható a javulás, inkább több Yagi antenna egymás mellé rakásával szoktak függönyantennát létrehozni.

Mivel a reflektor hosszabb mint a DE, ezért az impedanciája induktív lesz, a direktoroké (mivel ők pedig rövidebbek) kapacitív. Ez egy fázisprogresszióhoz fog vezetni az antenna mentén, amely hasonló lesz egy haladóhulláméhoz, a DE terét a direktorok irányába fogja erősíteni. 

A Yagi antenna rendszerre tekinthetünk egy olyan struktúraként, amely egy haladóhullámot támogat, amelynek a teljesítménye az egyes elemek árameloszlásán és a fázissebességen múlik \cite{balanis}.

\clearpage

\section{Tervezés}

A tervezés során a célom, hogy eredményképpen egy olyan antennát kapjak, ami tükrözi az \ref{fig:kathrein}. ábrát, tehát amelynek a VSWR-je <1.15, gain legalább 6dBi, bemeneti impedancia 50$\Omega$ és hasonló iránykarakterisztikája van.

A tervezés során a célom, hogy eredményképpen egy olyan antennát kapjak, amely tükrözi a(z) \ref{fig:kathrein}. ábrát, tehát:

\begin{itemize}
    \item \textbf{VSWR:} $< 1,15$.
    \item \textbf{Gain:} Legalább 6 dBi.
    \item \textbf{Bemeneti impedancia:} 50 $\Omega$.
    \item \textbf{Iránykarakterisztika:} A \ref{fig:kathrein}. ábrához hasonló.
\end{itemize}

Az adott értékek:
\begin{table}[ht]
    \centering
    \begin{tabular}{|lll|}
        \hline
        \textbf{Szimbólum} & \textbf{Leírás} & \textbf{Érték[m]} \\ \hline
        A & Reflektor és utolsó direktor távolsága & 2.95 \\
        B & Reflektor hossza & 2.51 \\
        C & Reflektorok távolsága Z síkban & 1.95 \\ \hline
    \end{tabular}
    \caption{Az adott dimenziók}
    \label{tab:dimenziok}
\end{table}

A szimulációt szabadtérben végeztem, a vezető anyagának rezet adtam meg, a kábelek sugarát 0.0025$\lambda$-nak választottam.

Parazitikus elemek jelenlétében a DE optimális hossza nem pontosan 0.5$\lambda$, hanem valahol 0.45-0.49$\lambda$ között van \cite{balanis}.

Bár az "Antenna Theory Analysis and Design" könyv azt írja, hogy tipikusan a direktorok közt 0.3-0.4$\lambda$ hely van, ez jelen esetben nem megvalósítható, mivel a reflektor és az utolsó direktor közti távolság adott, és túl nagy lenne a lépésköz. Más forrásokban azt találtam, hogy 0.25-0.005$\lambda$ közti távolságokat gyakran alkalmaznak \cite{yagi-video}.

Az optimális tervhez a direktorok közti távolság nem konstans. Kiindulásképpen beállítottam ezeket a távolságokat megfelelő kompromisszumnak tűnő értékekre.

Chen és Cheng kutatásán megihletődve ezután a 4nec2 optimizer-ének használatával optimalizáltam a távolságokat a direktorok között, először csak az SWR-re és a reaktanciára egyenlő súllyal. Ezután megadtam paraméternek a DE és a direktor hosszát is, a DE és a reflektor távolságát, és a DE-direktor illetve a direktorok távolságát, és előre-hátra viszonyra, SWR-re és reaktanciára egyenlő súlyokkal optimalizáltam még egyszer \cite{cheng}.

Az eredeti és az optimalizált értékeket a \ref{tab:yagi_parameters}. táblázat mutatja.

\begin{table}[ht]
    \centering
    \begin{tabular}{|llll|}
        \hline
        \textbf{Szimbólum} & \textbf{Leírás} & \textbf{Eredeti érték} & \textbf{Optimalizált érték} \\ \hline
        DE & Sugárzó hossza (Driven element) & 0.48$\lambda$ & 0.48$\lambda$ \\
        DirLen & Direktorok hossza & 0.45$\lambda$ & 0.45$\lambda$ \\ 
        RD & Reflektor--sugárzó távolság & 0.15$\lambda$ & 0.2$\lambda$ \\ 
        DE\_D1 & Sugárzó--D1 távolság & 0.15$\lambda$ & 0.15$\lambda$ \\ 
        D1\_D2 & D1--D2 távolság & 0.1$\lambda$ & 0.083$\lambda$ \\ 
        D2\_D3 & D2--D3 távolság & 0.08$\lambda$ & 0.054$\lambda$ \\ 
        D3\_D4 & D3--D4 távolság & 0.06$\lambda$ & 0.062$\lambda$ \\ \hline
    \end{tabular}
    \caption{A Yagi antenna eredeti és optimalizált paraméterei}
    \label{tab:yagi_parameters}
\end{table}

\FloatBarrier

Optimalizálás előtt a VSWR 4.45dB volt, utána 1.14, tehát jelentősen javult a geometriának az enyhe módosításával, anélkül, hogy a nyereséget befolyásolta volna.

\begin{figure}[ht]
     \centering
     % Első kép
     \begin{subfigure}[b]{0.48\textwidth}
         \centering
         \includegraphics[width=\textwidth]{main-58-pre-opt.png}
         \caption{Eredeti adatokkal mért eredmények}
         \label{fig:main-pre}
     \end{subfigure}
     \hfill % Rugalmas köz kitöltése a két kép között
     % Második kép
     \begin{subfigure}[b]{0.48\textwidth}
         \centering
         \includegraphics[width=\textwidth]{main-58.png}
         \caption{Optimalizált adatokkal mért eredmények}
         \label{fig:main}
     \end{subfigure}
     
     \caption{Közös felirat a két képnek (pl. Eredeti és optimalizált állapot)}
     \label{fig:osszehasonlitas}
\end{figure}

Balanis azt írta, hogy a reflektor és a DE távolsága főkéne az előre-hátra viszonyra és a bemeneti impedanciára van hatással, szóval kíváncsiságból minden mást változatlanul hagyva megváltoztattam a reflektor-DE távolságot 0.2$\lambda$-ra, és már így is 2.4-re javult a VSWR. 


\end{document}
